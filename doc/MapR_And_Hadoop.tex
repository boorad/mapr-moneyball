\documentclass[10pt,letterpaper]{article}
\usepackage[latin1]{inputenc}
\usepackage{amsmath}
\usepackage{amsfonts}
\usepackage{amssymb}
\usepackage{tikz}
\usepackage{graphicx}
\usepackage{caption}
\usepackage{subcaption}
\usepackage[top=1in,bottom=1in,right=1in,left=1in]{geometry}
\begin{document}
\begin{description}
\item[Hadoop] A platform for storage and processing of large data sets. Hadoop consists of a distributed filesystem, Java interfaces, and services for managing distribution of tasks. Core code is maintained by the Apache Software Foundation.
\item[MapR distribution for Hadooop] MapR provides an enterprise-ready distribution for Hadoop including the core open source components as well as a high-performance distributed filesystem, high-availability services, and easy-to-use management interface.
\item[MapReduce] A programming model for parallel processing, introduced in Google's 2004 paper\footnote{http://research.google.com/archive/mapreduce.html}. Hadoop provides a framework for running MapReduce jobs written in Java.
\item[Pig] A high level data flow language for processing data. Pig programs describe steps to be executed on sets of tuples, and are executed by running one or more MapReduce jobs.
\item[Hive] A framework for executing SQL-like queries on large data sets. The queries are executed as MapReduce jobs.
\end{description}
\end{document}
